\documentclass[a4paper,12pt]{report}
\usepackage[utf8]{inputenc}           % vagy latin2 helyett utf8
\usepackage[T1]{fontenc}              % karakterkódolás
\usepackage[hungarian]{babel}           % angol, magyar beállítások

%\usepackage[nottoc]{tocbibind}
%\usepackage{url}
\usepackage[hidelinks]{hyperref}
\hypersetup{colorlinks=false}
\urlstyle{same}

%\bibliographystyle{plain}

\usepackage{listings}                 % forráskód-beágyazás

\usepackage{multicol}
\usepackage{longtable}

\frenchspacing                % helyközök
%\usepackage{times}           % betűtípus
\usepackage{lmodern}          % vagy inkább ez

\usepackage[margin=2.5cm,left=3.5cm,includeheadfoot]{geometry}
                              % margók
\usepackage{graphicx}         % képekhez
\usepackage{setspace}         % sorköz
\onehalfspacing               % másfeles

\usepackage{pgfplots}
\pgfplotsset{width=7cm,compat=1.12}

\begin{document}

% ------------------------------------------------------------------------------
% Címlap


\begin{titlepage}

\noindent
\parbox[m]{0.2\textwidth}{
%\includegraphics[width=0.2\textwidth]{elte_cimer_ff.eps}     % fekete-fehér
 \includegraphics[width=0.2\textwidth]{images/elte_cimer_szines.eps} % színes
}
\hfill
\parbox[m]{0.7\textwidth}{
\begin{center}
\begin{large}
\textsc{
Eötvös Loránd Tudományegyetem\\
\vspace{0.5pc}
Informatikai Kar\\
\vspace{0.5pc}
Média- és Oktatásinformatikai tanszék\\
}
\end{large}
\end{center}
}

\vspace{1pc}
\hrule

\vfill

\begin{center}
{\LARGE Egy a mindennapokat megkönnyítő valós idejű okos otthon alkalmazás megvalósítása}
\end{center}

\vfill

\noindent
\hspace*{0.05\textwidth}
\parbox{0.45\textwidth}{
{\it Témavezető:}
\bigskip

{\Large Heizlerné Bakonyi Viktória }
\smallskip

Műszaki tanár
}
\hfill
\parbox{0.45\textwidth}{
{\it Szerző:}
\bigskip

{\Large Hegedüs Ádám}
\smallskip

Programtervező Informatikus BSc
}


\vfill

\begin{center}
{\large {\it Budapest, 2018}}
\end{center}

\end{titlepage}


% ------------------------------------------------------------------------------
% Témabejelentő

%\vspace*{\fill}
%\begin{center}
%Ehelyett az oldal helyett a Szakdolgozat-téma bejelentő szerepel.
%\end{center}
%\vfill
%\thispagestyle{empty}
%\newpage
\setcounter{page}{1}

% ------------------------------------------------------------------------------
% Tartalomjegyzék

\tableofcontents

% ------------------------------------------------------------------------------


\chapter{Bevezetés}

\section{Motiváció}
    Napjainkban mindent átsző a technnológia. Jelen van az oktatásban, egészségügyben, tömegközelekedésben, az élet rengeteg
    területén, de ami talán a legfontosabb, az otthonunkban. A legtöbb háztartásban előfordulnak okos eszközök, a család
    minden tagja ismerkedik a számítógép használatával, unokától kezdve a nagymamáig. De mégis mi célt szolgál a technológia?
    Szórakozás, információszerzés, kapcsolattartás távoli ismerősökkel, millió oka lehet annak hogy valaki laptopot ragadjon.
    Talán az egyik legfontosabb ezek közül az, hogy eszközeink segítsék, könnyebbé tegyék mindennapjainkat. Ez a gondolat
    motivált a dolgozatom megírására, szerettem volna egy olyan alkalmazást elkészíteni mely egyszerű funkciót tölbe, mégis
    hasznos lehet a dolgos hétköznapjainkon. Bárkivel előfordulhat hogy felkapcsolva felejti a lámpát a reggeli rohanás során
    és csak a munkahelyén jut eszébe. A szakdolgozatom erre a problémára szeretne megoldást kínálni. Az alkalmazás segítségével
    bárki regisztrálhat egy lámpát, akár több ember ugyanazt a fényforrást, és távolról fel -és lekapcsolhatja, követheti hogy
    mikor történt változás, illetve mennyi ideig volt összesen bekapcsolt állapotban az eszköz. Mindez valós időben történik,
    tehát minden felhasználó a legfrissebb információkkal rendelkezik a program használata közben.

\section{A programról}
    Az alkalmazás három részből áll, két kliensalkalmazásból, valamint egy szerverből. A Windows 10 klienst használhatják
    bejelentkezett felhasználók, ezen a kényelmes felületen találhatják a fő funckiókat, lámpát adhatnak hozzá profiljukhoz, valós időben
    vezérelhetik azt, megtekinthetik az eszköz statisztikáit. A másik kliensalkalmazást egy Raspberry Pi 3 mikroprocesszor futtatja, mely
    rá van kötve a vezérelni kívánt lámpára, szabályozza annak áramellátását, a szervertől kapott utasítások alapján. Minden változás után
    visszaigazolja a sikeres kapcsolást, valamint módosítások körülményeit is elküldi a szervernek, így az rögzítheti az eseményeket az
    adatbázisban. A szerver tehát köztes szerepet tölt be, a felhasználó rajta keresztül tud üzenni a eszközének, a lámpa pedig tőle
    kapja az utasításokat. Az adatbázis-kezelés is az ő feladata.

\begin{figure}[h!]
    \includegraphics[width=\linewidth]{images/application_diagram.jpg}
    \caption{Az alkalmazás struktúrája}
    \label{fig: Az alkalmazás struktúrája}
\end{figure}

    
% ------------------------------------------------------------------------------



\end{document}
